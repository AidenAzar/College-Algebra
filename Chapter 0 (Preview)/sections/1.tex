\chapter{Set and Numbers}
%%%%%%%%%%%%%%% SECTION HEADER %%%%%%%%%%%%%%%%
\rhead{1}
\lhead{Set and Numbers}
%%%%%%%%%%%%%%%%%%% START %%%$%%%%%%%%%%%%%%%%%
\section{Set}
A set is a collection of objects. For example, the set of days of the week is a set that contains 7 objects: Monday, Tuesday, Wednesday, Thursday, Friday, Saturday, and Sunday. The things in the collection are called \textbf{elements of the set}.
% ======= SUBSECTION
\subsection{Set Notation}
A set is often expressed by listing their elements between commas. enclosed by braces. We often let uppercase letters stand for sets. For example, set S containing numbers $1,\,2,\,3,\,4$ and 5 can be written as:
\[
S =\{1,\,2,\,3,\,4,\,5\}
\]
This  form  of  representation,  called  \textbf{the roster  method}.\\
A set can also be written in \textbf{set-builder notation}. In  this  notation,  the  elements  of  the  set  are  described  but  not  listed.  Here  is  an  example
\begin{gather*}
    \{\, \mathterm[a]{x} \mathterm[b]{\mid} x \mathterm[c]{\text{ is positive number less than 6}}\,\}
\end{gather*}
The set of all x \indicate{a}[out=0,in=-75] \\
\hspace*{2.5cm} such that \indicate{b}[out=0,in=-90] \hspace{0.5cm}
description \indicate{c}[out=0,in=-45]

\vspace{0.5cm}
The same set using the roster method is
\[
        \{1,\, 2,\, 3,\, 4,\, 5\}
\]
\subsection{Infinite and finite sets}
Consider set G. \[
        G= \left\{\frac{1}{2},\, \frac{1}{4},\, \frac{1}{8},\, \frac{1}{16},\,  \ldots\right\}
\]
Here the dots, called ellipsis, indicate a pattern of numbers that continues forever. A set is called an \textbf{infinite set} if it has infinitely many elements; otherwise it is called a \textbf{finite set}.
% ======= SUBSECTION
\subsection{Empty set}
There is a special set that, although small, plays a big role. The
empty set or Null set is the set $\{\}$ that has no elements. We denote it as $\phi$, so $\phi=\{\}$.
Whenever you see the symbol $\phi$, it stands for $\{\}$.
% ======= SECTION
\section{Numbers}
Some sets are so significant and prevalent that we reserve special
symbols for them. The set of natural numbers (i.e., the positive numbers) is denoted by $\mathbb{N}$, that is
\begin{equation*}
        \mathbb{N} = \{1,\,2,\,3,\,4,\, \ldots\}
\end{equation*}
Set of whole numbers, $\mathbb{W}$, is containing all natural numbers
including 0
\begin{equation*}
        \mathbb{W} = \{0,\,1,\,2,\,3,\,4,\,\ldots \}
\end{equation*}
The set of integers
\begin{equation*}
        \mathbb{Z} = \{\dots,\, -3,\,-2,\,-1,\,0,\,1,\,2,\,3,\,\ldots \}
\end{equation*}
is another fundamental set. This set not only includes positive and
0, but also negative numbers.\\
The set of rational numbers is the set of all numbers that can be
expressed as a quotient of two integers, with the denominator not 0.
\begin{equation*}
        \mathbb{Q} = \left\{\,\frac{a}{b}\, \mid\, \text{a and b are integers and $b\neq0$}\,\right\}
\end{equation*}
Needles to say, there is a set of irrational numbers $\mathbb{I}$. This set is the set of all numbers whose decimal representations are neither terminating nor repeating. Irrational numbers cannot be expressed as a quotient of integers.
\begin{gather*}
    \mathbb{I}=\{\, x \mid x \text{ is not a rational number}\,\}
\end{gather*}
Examples of irrational numbers are
\begin{align*}
        \sqrt{2} &\approx 1.414214 \\
        -\sqrt{3} &\approx -1.73205\\
        \pi &\approx 3.14159\\
        e &\approx  2.71828
\end{align*}
Finally, we have a set of real numbers which is the set if numbers
that are either rational or irrational; In other words, all numbers we are familiar with them.
\begin{equation*}
    \mathbb{R}=\{\, x \mid x \text{ is rational or irrational}\,\}
\end{equation*}
