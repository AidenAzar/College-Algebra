\chapter{Exponential and Logarithmic Equations}
%%%%%%%%%%%%%%% SECTION HEADER %%%%%%%%%%%%%%%%
\lhead{Exponential and Logarithmic Equations}
\rhead{4.4}
%%%%%%%%%%%%%%%%%%% START %%%$%%%%%%%%%%%%%%%%%
In this section, we will learn how to solve the exponential and logarithmic equations
An exponential is an equation containing a variable in an exponent. Examples are 
\[
	2^{(4x+1)}=8	\qquad		(12)^{(3x-x^2)}=144	 \qquad	(18)^5=18^{(y^3-12)}
\]
On the other hand, a logarithmic equation is an equation containing a variable in a 
logarithmic expression. Examples are
\[
	\log_{12}(y+4)=4	\qquad		\ln{(z+4)}-\ln{(2z-1)}=\ln{\frac{1}{z}}	 \qquad
	\log_{3}10=\log_{3}(x^2+7x)
\]
\section{Exponential equations}
There are two different types of exponential equations:
\begin{itemize}
 \item Exponential equations with the same base on each side;
 \item Exponential equations with the different bases.
\end{itemize}
\begin{tcolorbox}[title=Exponential equations, 
                  fonttitle=\bfseries,
                  colframe=green!70!black,
                  colback=green!60!white]
 \textbf{Exponential equations with the same base on each side}
	\[
	\text{if  } b^M=b^N \text{, then } M=N	
	\]
 \textbf{Exponential equations with different bases}
 	\begin{enumerate}
 		\item Isolate the exponential expression.
 		\item If the base is 10, then take common logarithm (log) on both sides
			  If the base is a number other than 10, take the natural logarithm (ln) on 
			  both sides.
		\item Apply power rule
		\item Solve for the variable
 	\end{enumerate}
\end{tcolorbox}
% ============ EXAMPLE 1
\begin{exa}
	Solve each equations.
	\begin{enumerate}[\bfseries a.]
	    \item $5^{3x-6} = 125$
	    \item $8^{x+2} =4^{x-3}$
	    \item $5^x =134$
	    \item $10^x = 8000$
	    \item $3^{2x-1} = 7^{x+1}$
	\end{enumerate}
%
a.	\begin{align*}
		5^{3x-6} = 125&		&&\text{125 is $5^3$}\\
		5^{3x-6} = 5&^3		&&\text{Same base on both sides}\\
		3x-6 = 3&			&&\text{Solve for $x$}\\
		x =3&				&&\text{Our Solution}\\
	\end{align*}
b.	\begin{align*}
		8^{x+2}=4^{x-3}&		&&\text{8 is $2^3$ and 4 is $2^2$}\\
		(2^3)^{x+2}=(2^2)^{x-3}&		&&\text{Use rule \eqref{prd_exp}}\\
		2^{3(x+2)}=2^{2(x-3)}&		&&\text{Same base on both sides}\\
		3(x+2)=2(x-3)&			&&\text{Solve for $x$}\\
		x =-12&					&&\text{Our solution}\\
	\end{align*}
c.	\begin{align*}
		5^x=134&				&&\text{Different bases, take ln}\\
		\ln{5^x}=\ln{134}&	&&\text{Apply power rule}\\
		x\ln{5}=\ln{134}&	&&\text{Solve for $x$}\\
		x = \frac{\ln{134}}{\ln{5}}&		&&\text{Use calculator}\\
		x \approx3.04&		&&\text{Our solution} \\
	\end{align*}
d.	\begin{align*}
	    10^x &= 8000			&&\text{Different bases, take log}\\
		\log_{}10^x &= \log_{}8000		&&\text{Apply power rule}\\
		x\log_{}10 &= \log_{}8000		&&\text{We know $\log_{}10=1$}\\
		x &= \log_{}8000					&&\text{Use calculator}\\
		x &\approx3.90					&&\text{Our Solution}
	\end{align*}
e.	\begin{align*}
	    3^{2x-1} = 7^{x+1}&				&&\text{Different bases, take ln}\\
		\ln{3^{2x-1}} =\ln{7^{x+1}}&		&&\text{Apply power rule}\\
		(2x-1)\ln{3} = (x+1)\ln{7}&		&&\text{Distribute}\\
		2x\ln{3}-\ln{3} = x\ln{7}+\ln{7}&		&&\text{To Isolate $x$, add $\ln{3}$}\\
		2x\ln{3}= x\ln{7}+\ln{3}+\ln{7}&		&&\text{Subtract $x\ln{7}$}\\
		2x\ln{3} -x\ln{7}=\ln{3}+\ln{7}&		&&\text{Factor $x$ on LHS}\\
		(2\ln{3}-\ln{7})x = \ln{3}+\ln{7}&		&&\text{Divide by $2\ln{3}-\ln{7}$}\\
		x = \frac{\ln{3}+\ln{7}}{2\ln{3}-\ln{7}}& 	&&\text{Use calculator}\\
		x \approx 12.11&			&&\text{Our solution}
	\end{align*}
\end{exa}
% ======= NOTE
\begin{nt}
Don't use a calculator immediately whenever you see a logarithm of a number in your 
equations. Instead, try to solve for an unknown variable and simplify as much as 
possible. Then use the calculator and round your answer.
\end{nt}
% ============ EXAMPLE 2
\begin{exa}
  Solve $e^{2x}-8e^{x}+7=0$.
 To solve these types of problems, first use \eqref{prd_exp} to rewrite $e^{2x}
 =(e^{x})^2$.  Then replace all $e^x$ with y,
 			\begin{align*}
 				(e^{x})^{2}-8e^x+7 &= 0 &&\text{Use \eqref{prd_exp}}\\
 				y^{2}-8y+7 &= 0  &&\text{Substitute $e^x=y$}
 			\end{align*}
 Using this trick, you can no easily see a trinomial which can be factored easily.
 $(y-7)(y-1)=0$. Then substitute back $y= e^{x}$.
 			\begin{align*}
 				\qquad (e^{x}-7)(e^{x}-1) &= 0 &&\text{Set each factor equal to zero}
 			\end{align*}
 So you will get
\begin{align*}
	\begin{cases}
		e^x-7=0 & \rightarrow e^x=7\\
		e^x-1=0 & \rightarrow e^x=1
	\end{cases}
\end{align*}
Take ln from both sides and recall from \eqref{ID_1} that $\ln{\,e}=1$
\begin{align*}
	\begin{cases}
		\ln{e^x}=7  \rightarrow x\ln{e}=7  \rightarrow x=7 \\
		\ln{e^x}=1  \rightarrow x\ln{e}=1  \rightarrow x=1
	\end{cases}
\end{align*}
\end{exa}
% ============ END EXAMPLE 2
\section{Logarithmic equations}
Similar to exponential equations, we have two situations:
\begin{itemize}
 \item	Logarithms with the same bases on both sides
 \item	Only one logarithm on one side
\end{itemize}
%
\begin{tcolorbox}[title=Logarithmic equations, 
                  fonttitle=\bfseries,
                  colframe=green!70!black,
                  colback=green!60!white]\textbf{Logarithms with the same bases on both sides}
\[
			\text{if } \log_{b}M=log_{b}N, \text{ then } M=N
					\]
\textbf{Only one logarithm on one side}
 	\begin{enumerate}
 		\item Isolate the log.
 		\item Use the definition of logarithms \eqref{log} rewrite the log in
 			  exponential form.
		\item Solve for variable.
 	\end{enumerate}
\textbf{Very important:}\\
\text{Always check your answer(s). The argument of log must be positive.}
\end{tcolorbox}
% =========== EXAMPLE 3
\begin{exa}
Solve the following logarithmic equation:
\[
            \ln{(x-3)}=\ln{(7x-23)}-\ln{(x+1)}
\]
\end{exa}
We start by condensing the LHS:
\begin{align*}
		\ln{(x-3)}=\ln{\frac{(7x-23)}{x+1}}&	&&\text{Same base on both sides}\\
		x-3 = \frac{(7x-23)}{x+1}&		&&\text{Multiply both sides by$(x+1)$}\\
		(x-3)(x+1) = 7x-23&			&&\text{FOIL} \\
		x^2-2x-3 =7x-23&			&&\text{subtract $7x-23$}\\
		x^2-9x-20=0&						&&\text{Factor}\\
		(x-4)(x-5)=0&						&&\text{Set each equal to zero}\\
		x=4\,\, \text{and}\,\, x=5&			&&\text{Always check the solutions} 
\end{align*}
Check.\\
Plug $x=4$ and $x=5$. If you get any negative log, then that answer is not correct. 
Otherwise, that $x$-value will be our answer. 
	\begin{align*}
		\text{For}\,\, x=4:  \qquad \qquad&\\
		\ln{(4-3)} =\ln{(7(4)-23)}-\ln{(4+1)}& &&\text{Plug it and then simplify} \\
		\ln{1}=\ln{(5)}-\ln{(5)}&				&&\text{No negative log appeared}\\
		x=4& \quad \checkmark					&&\text{Correct answer}
		&&&\\
		\text{For}\,\, x=5: \qquad \qquad&  \\
		\ln{(5-3)}=\ln{(7(5)-23)}-\ln{(5+1)}& &&\text{Plug it and then simplify} \\
		\ln{2}=\ln{(12)}-\ln{(6)}&			&&\text{No negative log appeared}\\
		x=5& \quad \checkmark					&&\text{Correct answer}		
	\end{align*}
The solution set is $\{4,\, 5\}$.	
% ============== EXAMPLE 4
\begin{exa}
    Solve the following logarithmic equations. 
    \[
               \log_{2}(x-2) =3
    \]
\end{exa}
	\begin{align*}
		\log_{2}(x-2) =3&		&&\text{One logarithm, use definition of log}\\
		2^3 = x-2&				&&\text{Solve for $x$}\\
		10 = x&					&&\text{Check the answer!}
	\end{align*}
check.	
	\begin{align*}
		\text{For}\,\, x=10:  \\
		\log_{2}(10-2) =3& 			&&\text{Plug it and then simplify} \\
		\log_{2}(8) =3& 				&&\text{No negative log appeared}\\
		x=10& \quad \checkmark		&&\text{Correct answer}
	\end{align*}
The solution set is $\{10\}$.	
% =============== EXAMPLE 5
\begin{exa}
    Solve the following logarithmic equations. 
    \[
                4\ln{(3x)} =8
    \]
\end{exa}
We need to isolate ln first,
	\begin{align*}
		4\ln{(3x)} =8&		&&\text{Divide both sides by 4}\\
		\ln{(3x)}  =2&			&&\text{One logarithm, use definition of log}\\
		3x = e^2&				&&\text{Solve for $x$}\\
		x= \frac{e^2}{3}&		&&\text{Check the answer!}
	\end{align*}
check.	
	\begin{align*}
		\text{For}\,\, x=\frac{e^2}{3}:  \\
		4\ln{\biggl(3\Bigl(\frac{e^2}{3}\Bigr)\biggr)} =8& 		&&\text{Plug it and then simplify} \\
		4\ln{(e^2)} =8&				&&\text{No negative log appeared}\\
		x=\frac{e^2}{3}& \quad \checkmark		&&\text{Correct answer}
	\end{align*}
The solution set is $\bigl\{\frac{e^2}{3}\bigr\}$.
% =============== EXAMPLE 6
\begin{exa}
    Solve the following logarithmic equations. 
    \[
                \log_{}x+\log_{}(x-3) =1
    \]
\end{exa}
	\begin{align*}
		\log_{}x+\log_{}(x-3) =1&	&&\text{Condense logs using \eqref{product}}\\
		\log_{}x(x-3) =1&			&&\text{One logarithm, use definition of log}\\
		x(x-3)= 10^1&				&&\text{Distribute LHS}\\
		x^2-3x=10&                  &&\text{Subtract 10 from both sides} \\
		x^2-3x-10 =0& 				&&\text{Factor}\\
		(x+2)(x-5) =0&					&&\text{Set each equal to zero}\\
		x=-2\,\, \text{and}\,\, x=5&	&&\text{Always check!} 
	\end{align*}
check.	
	\begin{align*}
		\text{For}\,\, x=-2:  \\
		\log_{}(-2)+\log_{}(-2-3) =1& 		&&\text{Plug it and then simplify} \\
		\log_{}(-2)+\log_{}(-5) =1& 				&&\text{Negative log appeared}\\
		x=-2& \quad \xmark		&&\text{Wrong answer}\\
		&&&\\
		\text{For}\,\, x=5:  \\
		\log_{}(5)+\log_{}(5-3) =1& 		&&\text{Plug it and then simplify} \\
		\log_{}(5)+\log_{}(2) =1 &		&&\text{No negative log appeared}\\
		x=5& \quad \checkmark		&&\text{Correct answer}\\	
	\end{align*}
The solution set is $\{5\}$.
