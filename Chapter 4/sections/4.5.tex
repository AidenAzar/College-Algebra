\chapter{Exponential Growth and Decay: Modeling Data}
%%%%%%%%%%%%%%% SECTION HEADER %%%%%%%%%%%%%%%%
\lhead{Exponential growth and decay: modeling data}
\rhead{5}
%%%%%%%%%%%%%%%%%%% START %%%$%%%%%%%%%%%%%%%%%
There are many phenomena in the worlds that the amount of something whether grows or decay fast. For example, compounding money, bacteria growth in a petri dish and radioactive decay of an element.
% ============
\section{The exponential growth}
An exponential growth process is one in which the rate of increase of a quantity is 
proportional to the present value of that quantity. The simplest example is a savings 
account.
	\begin{tcolorbox}[title=The exponential growth, fonttitle=\bfseries,
	                  colframe=blue!70!black,
	                  colback=white]
		The model for exponential growth is 
		\begin{equation}
				A=A_0e^{kt} \label{exp_growth}	
		\end{equation}
		where,
		\begin{align*}
			A_0&= \text{initial amount or size}\\
			t &=  \text{time} \\
			A &= \text{The amount at time $t$}\\
			k &= \text{a constant.}
		\end{align*}
		In this model, $k>0$ and it is called the growth rate.
	\end{tcolorbox}
% =============== EXAMPLE 1
\begin{exa}
In 2000, the population of Africa was 807 million and by 2011 it had grown to 1052
million. 
	\begin{enumerate}[a.]
	 \item Use the exponential growth model $A = A_0 e^{kt}$ , in which $t$ is the number of years after 2000, to find the exponential growth function that models the data.
	 \item By which year will Africa’s population reach 2000 million, or two billion?
	\end{enumerate}
a. We are given 
		\begin{align*}
				A_0&=807 \quad \text{million}\\
				t &=11 \quad \text{years}\\
				A &= 1050 \quad \text{million}
		\end{align*}
and we are looking for $k$. Plug them into $A = A_0 e^{kt}$ and solve for $k$.
		\begin{align*}
			A &= A_0e^{kt} &&\text{Substitute $A$, $A_0$ and $t$}\\
			1052 &= 807e^{k(11)} &&\text{Divide both sides by 807}\\
			\frac{1052}{807} &= e^{k(11)} &&\text{Exponential equation, take ln}\\
			\ln{\left(\frac{1052}{807}\right)} &= \ln{e^{k(11)}} &&\text{Apply power rule}\\
			\ln{\left(\frac{1052}{807}\right)} &=k(11)\ln{e} &&\text{We know $\ln{e}=1$}\\
			\ln{\left(\frac{1052}{807}\right)} &=k(11) &&\text{Divide both sides by 11}
				\\
			\frac{\ln{\left(\frac{1052}{807}\right)}}{11} &=k &&\text{Use calculator}\\
			\frac{0.2651247}{11} &= k && \\
			0.024 &\approx k && \text{Our solution}	
		\end{align*}
b. We have
		\begin{align*}
				A_0&=807 \quad \text{million}\\
				A &= 2000 \quad \text{million}
		\end{align*}
and we are looking for $t$. Substitute and solve for it.
		\begin{align*}
			2000 &= 807e^{0.024t} &&\text{Divide both sides by 807}\\
			\frac{2000}{807} &= e^{0.024t} &&\text{Exponential equation, take ln}\\
			\ln{\left(\frac{2000}{807}\right)} &= \ln{e^{0.024t}} &&\text{Apply power rule}\\
			\ln{\left(\frac{2000}{807}\right)} &=0.024t\ln{e} &&\text{We know $\ln{e}=1$}\\
			\ln{\left(\frac{2000}{807}\right)} &=0.024t &&\text{Divide both sides by 0.024}\\
			\frac{\ln{\left(\frac{2000}{807}\right)}}{0.024} &=t &&\text{Use calculator}\\
			\frac{0.024102248}{0.024} &= t && \\
			38 &\approx t && \text{Our solution}	
		\end{align*}
So after 38 years, in 2038, the population of Africa reach 2000 million.
\end{exa}
% ========== SECTION
\section{The exponential decay}
When a population decays exponentially, it decreases at a rate that is proportional to 
its size at any time $t$.
	\begin{tcolorbox}[title=The exponential decay, fonttitle=\bfseries,
	                  colframe=blue!70!black,
	                  colback=white]
		The model for exponential decay is 
		\begin{equation}
				A=A_0e^{kt} \label{exp_decay}	
		\end{equation}
		where,
		\begin{align*}
			A_0&= \text{initial amount or size}\\
			t &=  \text{time} \\
			A &= \text{The amount at time $t$}\\
			k &= \text{a constant.}
		\end{align*}
		In this model, $k<0$ and it is called the decay rate.
	\end{tcolorbox}
\begin{nt}
Many times, the amount of a substance is expressed in terms of half-life, meaning
	\[ 
		A=A_0/2
	\]
the time it takes for half of any given quantity to decay so that only half of its 
original amount remains. 
\end{nt}
% =========== Example 2
\begin{exa}
Strontium-90 is a waste product from nuclear reactors. As a consequence of fallout from 
atmospheric nuclear tests, we all have a measurable amount of strontium-90 in our bones.
	\begin{enumerate}[a.]
		\item The half-life of strontium-90 is 28 years, meaning that after 28 years a given amount of the substance will have decayed to half the original amount. Find the exponential decay model for strontium-90. 
		\item Suppose that a nuclear accident occurs and releases 60 grams of strontium-90 into the atmosphere. How long will it take for strontium-90 to decay to a level of 10 grams?
	\end{enumerate}	
\newpage
a. We know the half-life of strontium-90 is 28 years. This means after 28 years, the
amount of strontium-90 will be $A=\frac{A_0}{2}$. Plug them into the \eqref{exp_decay} 
and solve for k. 
		\begin{align*}
			A = A_0e^{kt}& &&\text{Substitute $A$, and $t$}\\
			\frac{A_0}{2} = A_0e^{k(28)}& &&\text{Cancel out $A_0$}\\
			\frac{1}{2} = e^{k(28)}& &&\text{Exponential equation, take ln}\\
			\ln{\left(\frac{1}{2}\right)}= \ln{e^{k(28)}}& &&\text{Apply power rule}\\
			\ln{\left(\frac{1}{2}\right)}=k(28)\ln{e}& &&\text{We know $\ln{e}=1$}\\
			\ln{\left(\frac{1}{2}\right)}=k(28)& &&\text{Divide both sides by 28}\\
			\frac{\ln{\left(\frac{1}{2}\right)}}{28}=k& &&\text{Use calculator}\\[.2cm]
			\frac{-0.6931472}{28}= k& && \\[.2cm]
			-0.0248 \approx k& && \text{Our solution}	
		\end{align*}
\end{exa}
b. We are looking for $t$, when $A_0=60$ grams and $A=10$ grams. Thus,
		\begin{align*}
			10 = 60e^{-0.0248t}& &&\text{Divide both sides by 10}\\
			\frac{10}{60}= e^{-0.0248t}& &&\text{Reduce LHS }\\
			\frac{1}{6}= e^{-0.0248t}& &&\text{Exponential equation, take ln}\\
			\ln{\frac{1}{6}}= \ln{e^{-0.0248t}}& &&\text{Apply power rule}\\
			\ln{\frac{1}{6}}=-0.0248t\ln{e}& &&\text{We know $\ln{e}=1$}\\
			\ln{\frac{1}{6}}= -0.0248t& &&\text{Divide both sides by -0.0248}\\[.3cm]
			\frac{\ln{\frac{1}{6}}}{-0.0248}=t& &&\text{Use calculator}\\[.3cm]
			\frac{-1.7917595}{-0.0248}=& t && \\[.3cm]
			72 \approx t& && \text{Our solution}	
		\end{align*}