\chapter{Combination of Functions}
%\addcontentsline{toc}{chapter}{1 Graphs}
%%%%%%%%%%%%%%% SECTION HEADER %%%%%%%%%%%%%%%%
\rhead{6}
\lhead{Combination of Functions}
%%%%%%%%%%%%%%%%%%% START %%%$%%%%%%%%%%%%%%%%%
\section{Domains of a function}
The domain of a function is the set of all real numbers you can substitute into the function; In other words, all $x$-values.\\
Consider the standard quadratic function, $f(x) = x^2$. As you can see, we can substitute any real number we want into this function; We can use $x=3$ or $-10$ or $10000$ or $3.14159$ or whatever and we still get an answer. So because there is no restriction on $x$, the domain is all real numbers. We can express it both in set-builder and interval notation as follows:
\begin{align*}
        \text{Set-builder: Domain} &: \{x \mid x\ \in \mathbb{R}\ \}\\
        \text{Interval: Domain} &: (-\infty,\, +\infty)
\end{align*}
Now let's consider $f(x)=\frac{1}{x-6}$. It looks like any number will work. But wait! What happens if we use $x=6$? The denominator becomes zero and we divide by zero, which is not allowed. Since any other number is fine, the domain of this function is all real number except $6$.  Therefore,
\begin{align*}
        \text{Set-builder: Domain} &= \{x \mid x\neq 6\ \}\\
        \text{Interval: Domain} &= (-\infty,\,6) \cup (6,\,+\infty)
\end{align*}
As you can see, there are some situation that impose a restriction on our domain. When determining the domain of a function, we really only have to look out for four situations:
\begin{enumerate}[label=\protect\circled{\arabic*}]
    \item \textbf{Rational Expressions}: Division by zero is not allowed. So we must omit any values of $x$ which make the denominator 0; In other words, the domain will be all real number except roots of denominator.
            \begin{equation*}
               f(x) =\frac{u}{v}\quad \longrightarrow \quad 
               \text{Domain}: \{x \mid v\neq 0\ \}\\
        \end{equation*}
    \item \textbf{Even roots}: For even roots such as square roots, the radicand can not be negative. Thus, to find the domain find all $x$ that makes the radicand non-negative: 
        \begin{equation*}
                f(x) =\sqrt{u} \quad \longrightarrow \quad 
               \text{Domain}: \{x \mid u \ge 0\ \}\\
        \end{equation*}
    \item \textbf{Even roots in the denominator}: In this case, since we have a even root, the radicand should be non-negative. However, it cannot be zero because it is located in denominator. Therefore, to find domain:
            \begin{equation*}
                f(x) =\frac{u}{\sqrt{v}} \quad \longrightarrow \quad 
               \text{Domain}: \{x \mid v > 0\ \}\\
        \end{equation*}
    \item \textbf{Logarithmic function}: In chapter 4, we'll see that logarithms can only be take of positive values.
\end{enumerate}
% ======= EXAMPLE 1
\begin{exa}
    Find the domain of each functions.
    \begin{enumerate}[\bfseries a.]
        \begin{multicols}{2}
        \item $f(x) = \frac{5x}{x^2-49}$
        \item $g(x) =x^2-4x+2$
        \item $ h(x) = \sqrt{9x-27}$
        \item $k(x)=\frac{5x}{\sqrt{24-3x}}$
        \end{multicols}
    \end{enumerate}
\end{exa}
\textbf{a.} We have a rational expression, so we must exclude roots of denominators. 
\begin{align*}
        x^2-49=0&   &   &\text{Set the denominator to 0, add 49}\\
        x^2=49& &   &\text{Take square root, don't forget $\pm$}\\
        x = \pm \sqrt{49}&  &   &\text{49 is a perfect square}\\
        x = \pm 7&  &   &\text{These roots are excluded}
\end{align*}
So the domain is all real numbers except $-7$ and $7$ so domain is \[(-\infty,-7) \cup (-7,7) \cup (7,+\infty)\]
\\
\textbf{b.} Since we don't have any type of restrictions, the domain is all real numbers or $(-\infty, +\infty)$.
\\[1cm]
\textbf{c.} The radicand of square roots should be non-negative, so
\begin{align*}
        9x - 27 \ge 0 &     &   &\text{Add 27}\\
        9x \ge 27&  &   &\text{Divide both sides by 9}\\
        x \ge 3&    &   &\text{Our domain}
\end{align*}
The domain is $x\ge 3$ or in interval notation $[3,+\infty)$.
\\[1cm]
\textbf{d. } The radicand should be non-negative but cannot be zero so
\begin{align*}
        24-3x > 0&  &   &   \text{Subtract 24}\\
        -3x > -24&  &   &\text{Divide both sides by $-3$, flip the sign!}\\
        x < 8&  &   &\text{Our domain}
\end{align*}
Thus, domain is $x < 8$ or using interval notation $(-\infty,8)$.
% ========= END Example
\section{The algebra of functions}
We can combine the functions using addition, subtraction, multiplication and division.
\begin{tcolorbox}[
                    title=Algebra of functions, fonttitle=\bfseries,
                    colframe=blue!70!red,
                    colback=white]
Let $f$ and $g$ be two functions:
\begin{align*}
    &\textbf{1. Sum:} &     &(f+g)(x) = f(x)+g(x)\\
    &\textbf{2. Difference:} &  &(f-g)(x)= f(x) - g(x)\\
    &\textbf{3. Product:} & &(fg)(x) = f(x)\cdot g(x)\\
    &\textbf{4. Quotient:} &    &\biggl(\frac{f}{g} \biggr)(x)= \frac{f(x)}{g(x)}
\end{align*}
\end{tcolorbox}
% ========= EXAMPLE 
\begin{exa}
    Let $f(x) =x-5$ and $g(x) = x^2-1$. Find each of the following functions:
    \begin{enumerate}[\bfseries a.]
    \begin{multicols}{2}
        \item $(f+g)(x)$
        \item $(f-g)(x)$
        \item $(fg)(x)$
        \item $\biggl(\frac{f}{g} \biggr)(x)$
    \end{multicols}
    \end{enumerate}
\end{exa}
%%
\vspace{0.5cm}
%%
\textbf{a. }
$(f+g)(x)$ is $f(x)+g(x)$ so,
\begin{align*}
    f(x)+g(x)&   &   &\text{Substitute}\\
    (x-5) + (x^2-1)&    &   &\text{Combine like terms}\\
    x+x^2-6&    &   &\text{Our solution}
\end{align*}
\\[0.2cm]
%%
\textbf{b. }
$(f+g)(x)$ is $f(x)-g(x)$ so,
\begin{align*}
    f(x)-g(x)&   &   &\text{Substitute}\\
    (x-5) - (x^2-1)&    &   &\text{Distribute $-1$}\\
    x-5-x^2+1&    &   &\text{Combine like terms}\\
    x-x^2-4&    &   &\text{Our solution}
\end{align*}
\\[0.2cm]
%%
\textbf{c. }
$(fg)(x)$ is $f(x)\cdot g(x)$ so,
\begin{align*}
    f(x)\cdot g(x)&   &   &\text{Substitute}\\
    (x-5)\cdot (x^2-1)&    &   &\text{FOIL method}\\
    x^3-x-5x^2+5&    &   &\text{Our solution}
\end{align*}
\\[0.2cm]
%%
\textbf{d. }
$\biggl(\frac{f}{g} \biggr)(x)$ is $\frac{f(x)}{g(x)}$ so,
\begin{align*}
    \frac{f(x)}{g(x)}&   &   &\text{Substitute}\\
    \frac{x-5}{x^2-1}&    &   &\text{Our solution}
\end{align*}
% ===========
\subsection{Domain of combining functions}
Domain of  each of these functions consists of all real numbers that are common to domains of $f$ and $g$. If $D_f$ represent domain of $f$ and $D_g$ represent domain of $g$, therefore the domain for each function is $D_f \cap D_g$. For the quotient case, we have to add the condition that denominator cannot be zero.
\begin{tcolorbox}[
                    title=Domain of combining functions, fonttitle=\bfseries,
                    colframe=blue!70!red,
                    colback=white]
Let $f$ and $g$ be two functions:
\begin{align*}
    &\textbf{1. Sum:} &     &\text{Domain: } D_f \cap D_g\\
    &\textbf{2. Difference:} &  &\text{Domain: } D_f \cap D_g\\
    &\textbf{3. Product:} & &\text{Domain: } D_f \cap D_g\\
    &\textbf{4. Quotient:} &    &\text{Domain: } D_f \cap D_g\quad  \text{and} \quad g(x)\neq 0\\
\end{align*}
\end{tcolorbox}
% ====== EXAMPLE 
\begin{exa}
    If $f(x)=5x+1$ and $g(x)=x^2-3x-4$, determine the domain of each of the following functions:
        \begin{enumerate}[\bfseries a.]
    \begin{multicols}{2}
        \item $(f+g)(x)$
        \item $(f-g)(x)$
        \item $(fg)(x)$
        \item $\biggl(\frac{f}{g} \biggr)(x)$
    \end{multicols}
    \end{enumerate}
\end{exa}
Since the equations for $f$ and $g$ do not involve division or contain even roots, the domain of both of them is the set of all real numbers. Therefore, the domain of $f+g$, $f-g$, and $fg$ is the set of all real numbers or $(-\infty,\,+\infty)$.\\
The function $\frac{f}{g}$ contains division. We must exclude the roots of denominator, $x^2-3x-4$. Let's find them
\begin{align*}
    x^2-3x-4=0&     &       &\text{Factor out}\\
    (x+1)(x-4)=0&   &       &\text{Set each factor equal to 0}\\
    x=-1\quad x=4&  &       &\text{roots of denominators}
\end{align*}
We must exclude $-1$ and $4$ from the domain of $\frac{f}{g}$. 
\begin{equation*}
    \text{Domain of }\frac{f}{g} = (\infty,-1) \cup (-1,4) \cup (4,+\infty)
\end{equation*}
% ========= SECTION
\section{Composite functions}
Another way to combine two functions is composition. A composition
of functions is a function inside of a function. The notation used for composition of functions is:
\begin{equation}
        (f\circ g)(x) = f(g(x))
\end{equation}
To find a composite function, we will take the inside function and substitute into the outside function.
% ===== EXAMPLE
\begin{exa}
    If $f(x) =x^2-3$ and $g(x) = x+2$, find each composite function:
    \begin{enumerate}[\bfseries a.]
        \item $(f \circ g)(x)$
        \item $(g \circ f)(x)$
    \end{enumerate}
\end{exa}
%
\vspace{0.2cm}
a. $(f \circ g)(x)$ means $f(g(x))$:
\begin{align*}
        f(\textcolor{red}{g(x)})&   &   &\text{Substitute $g(x)$ into $f(x)$}\\
        (\textcolor{red}{g(x)})^2-3&    &   &\text{$g(x)$ is $x+2$}\\
        (\textcolor{red}{x+2})^2-3& &   &\text{Simplify}\\
        x^2+2^2+2(x)(2) -3& &   &\text{Combine like terms, $4$ and $-3$}\\
        x^2+4x+1&   &   &\text{Our solution}
\end{align*}
%
\\[0.5cm]
%
b. $(g \circ f)(x)$ means $g(f(x))$:
\begin{align*}
        g(\textcolor{red}{f(x)})&   &   &\text{Substitute $f(x)$ into $g(x)$}\\
        \textcolor{red}{f(x)}+2&    &   &\text{$f(x)$ is $x^2-3$}\\
        \textcolor{red}{x^2-3}+2& &   &\text{Combine like terms}\\
        x^2-1& &   &\text{Our solution}
\end{align*}
% ======= NOTE 1
\begin{nt}
    It is important to note that very rarely is $(f \circ g)(x)$ the same as $(g \circ f)(x)$ as the previous example showed.
\end{nt}
% ======== NOTE 2
\begin{nt}
    To calculate a composition of function we will evaluate the inner function and
substitute the answer into the outer function. This is shown in the following
example.
\end{nt}
% ======== EXAMPLE 
\begin{exa}
    If $f(x) =x^2-2x+1$ and $g(x) = x-5$, evaluate each composite function:
    \begin{enumerate}[\bfseries a.]
        \item $(f \circ g)(3)$
        \item $(g \circ f)(-1)$
    \end{enumerate}
\end{exa}
%
\vspace{0.2cm}
a. $(f \circ g)(3)$ is actually $f(g(3))$. Evaluate the inner function first, $g(3)$:
\begin{align*}
        g(x) &= x-5  &   &\text{This is $g(x)$}\\
        g(3) &= 3-5  &   &\text{Subtract}\\
             &= -2   &   &\text{This is $g(3)$}
\end{align*}
Now $-2$ goes into $f$,
\begin{align*}
    f(-2) &= (-2)^2-2(-2)+1     &       &\text{Simplify}\\
          &= 4+4+1              &       &\text{Add}\\
          &= 9                  &       &\text{Our solution}
\end{align*}
%
\\[0.2cm]
%
b.  $(g \circ f)(-1)$ is actually $g(f(-1))$. Evaluate the inner function first, $f(-1)$:
\begin{align*}
        f(x)  &= x^2-2x+1  &   &\text{This is $f(x)$}\\
        f(-1) &= (-1)^2-2(-1)+1  &   &\text{Simplify and add}\\
              &= 4   &   &\text{This is $f(-1)$}
\end{align*}
Now $4$ goes into $g$,
\begin{align*}
    g(4) &= 4-5      &       &\text{Simplify}\\
         &= -1       &       &\text{Our solution}\\
\end{align*}
% ========== SUB
\subsection{Domain of a composite function}
Finding the domain of a composite function consists of two steps: 
\begin{enumerate}[\bfseries 1.]
    \item Find the domain of the "inside" (input) function. If there are any restrictions on the domain, keep them. 
    \item Construct the composite function. Find the domain of this new function. If there are restrictions on this domain, add them to the restrictions from Step 1. 
\end{enumerate}
% ====== EXAMPLE
\begin{exa}
    Given $f(x)= x^2+2$ and $g(x)=\sqrt{7-x}$, find each of the following:
    \begin{enumerate}[\bfseries a.]
    \begin{multicols}{2}
        \item $(f \circ g)(x)$
        \item Domain of $(f \circ g)(x)$
    \end{multicols}
    \end{enumerate}
\end{exa}
%
\vspace{0.2cm}
%
a. $(f \circ g)(x)$ is $f(g(x))$ so
\begin{align*}
    f(\textcolor{red}{g(x)})&    &   &\text{$g(x)$ goes into $f(x)$}\\
    (\textcolor{red}{g(x)})^2+2& &   &\text{Replace $g(x)$ with $\sqrt{7-x}$}\\
    (\textcolor{red}{\sqrt{7-x}})^2+2&   &   &\text{Simplify}\\
    7-x +2 &    &   &\text{Combine like terms}\\
    9-x&    &   &\text{Our solution}
\end{align*}
%
\\[0.2cm]
%
b. Domain of $(f \circ g)(x)$:
\begin{enumerate}[1.]
    \item We must find domain of inside, $g(x)$. Since $g(x)$ contains even roots, therefore
            \begin{align*}
                7-x \ge 0&  &   &\text{Solve for $x$}\\
                x \le 7&    &   &\text{Keep this!}
            \end{align*}
    \item In next step, we must look at the composite function itself. Since $(f \circ g)(x) = 9-x$, This function puts no additional restrictions on the domain. So the composite domain is $x \le 7$.
\end{enumerate}
%
% ======== EXAMPLE 
\begin{exa}
    Given $f(x) = \frac{3x}{x-1}$ and $g(x) = \frac{2}{x}$, find each of the following:
    \begin{enumerate}[\bfseries a.]
    \begin{multicols}{2}
        \item $(f \circ g)(x)$
        \item Domain of $(f \circ g)(x)$
    \end{multicols}
    \end{enumerate}
\end{exa}
%
\vspace{0.2cm}
%
a. $(f \circ g)(x)$ is $f(g(x))$ so
\begin{align*}
    f(\textcolor{red}{g(x)})&    &   &\text{$g(x)$ goes into $f(x)$}\\
    \frac{3\,\textcolor{red}{g(x)}}{\textcolor{red}{g(x)}-1}& &   &\text{Replace $g(x)$ with $\frac{2}{x}$}\\
    %
    \frac{3\textcolor{red}{(\frac{2}{x})}}{\textcolor{red}{(\frac{2}{x})}-1}&   &   &\text{Simplify by multiply by $\frac{x}{x}$}\\
    \frac{(\frac{6}{x})}{(\frac{2}{x})-1}\cdot \bm{\frac{x}{x}}&    &   &\text{Multiply}\\
    \frac{6}{2-x}&    &   &\text{Our solution}
\end{align*}
%
\\[0.2cm]
%
b. To find domain of $(f \circ g)(x)$, follow these steps:
\begin{enumerate}[1.]
    \item We must find domain of inside, $g(x)$. Since $g(x)$ contains division, therefore $x\ne 0$. Keep this!
    \item We must find the domain of $(f \circ g)(x) = \frac{6}{2-x}$. Since this function also contains division, $2-x\neq 0$, gives us $x\neq 2$.
\end{enumerate}
Combine this domain with the domain from Step 1: the composite domain is $\{x \mid x\neq 0 \text{ and } x\neq 2\}$ or in interval notation $(-\infty,0) \cup (0,2) \cup (2,+\infty)$.
