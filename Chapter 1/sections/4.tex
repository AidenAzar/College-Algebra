\chapter{Complex Numbers}
%\addcontentsline{toc}{chapter}{7.6 Imaginary Numbers}
% SECTION HEADER
\rhead{4}
\lhead{Complex Numbers}
%
\section{Imaginary unit}
When mathematics was first used, the primary purpose was
for counting. Thus they did not originally use negatives, zero, fractions or irrational
numbers. However, the ancient Egyptians quickly developed the need for “a
part” and so they made up a new type of number, the ratio or fraction. The
Ancient Greeks did not believe in irrational numbers (people were killed for
believing otherwise). The Mayans of Central America later made up the number
zero when they found use for it as a placeholder. Ancient Chinese Mathematicians
made up negative numbers when they found use for them. \\
The square roots of a negative number created enough
of problem to stop many mathematician in their tracks. Square roots ask
us to find a number, that when multiplied by itself, yield
the number inside the root sign. The square root of nine is
three because three times three is nine.
\[ \sqrt{9}=3 \]
But what about roots of negative numbers? What is the
square root of negative nine? Positive three won’t work,
and neither will negative three, so we’re stuck. 
\[
    \sqrt{-9}=? 
\]
Usually mathematicians would interpret this as the problem’s way of saying there
are no solutions. Like 0 and negative numbers before, $\sqrt{-1}$ was generally regarded 
with suspicion because it did not correspond to anything people could think of it in the real
world. For this reason, $\sqrt{-1}$ was given the terrible names "imaginary" or "impossible".

\vspace{0.4cm}
A century or so later, Euler begin using the simple $i$ to indicate that $\sqrt{-1}$ exist,
making the algebra less clunky. Unfortunately, the name imaginary stuck around and that's still
we call these numbers today. In response, everything on the original number line gets the name
"real". 
	%
	\begin{tcolorbox}[title={Imaginary unit $i$},fonttitle=\bfseries,
	colframe=blue!70!red,
	colback=white]
	\begin{equation}
	 i=\sqrt{-1},\,\text{thus} \quad i^2=-1 \label{img_num}
	\end{equation}
	\end{tcolorbox}
	%
Using this notation, we can express any negative radicand in terms of $i$
\begin{equation*}
    \sqrt{-a}=\sqrt{-1}\sqrt{a}=i\sqrt{a} 
\end{equation*}
\begin{tcolorbox}[title=
                    Principle square root of a negative number,
                    fonttitle=\bfseries,
                    colframe=blue!70!red,
	                colback=white]
    \begin{equation}
            \sqrt{-a}=i\sqrt{a} 
             \label{break_i}
    \end{equation}
\end{tcolorbox}
%============= Example 1
\begin{exa}
 \begin{align*}
	a) \sqrt{-49} &=\sqrt{-1}\sqrt{49} =i\sqrt{49} = 7i &&&&\\
	b) \sqrt{-31} &=\sqrt{-1}\sqrt{31} =i\sqrt{31}&&&&\\
	c) \sqrt{-98} &=\sqrt{-1}\sqrt{98} =i\sqrt{98}=i\sqrt{2\cdot7^2}=7i\sqrt{2} &&&&\\
 \end{align*}
\end{exa}
%============= END
\begin{nt}
	We cannot join two square roots using product rule, $\sqrt{a}\cdot \sqrt{b}=\sqrt{ab}$, when a and b are
	both negative numbers. To multiply negative radicands, first use the definition of imaginary
	number to take out $i$. Then multiply the radical with positive radicands. For example:
	\begin{align*}
	\sqrt{-8}\cdot\sqrt{-2}=i\sqrt{8}\cdot i\sqrt{2}&=i^2\sqrt{16}\\
	&=(-1)\sqrt{4^2}\\
	&=-4
	\end{align*}
\end{nt}
% ===========
\section{Complex Numbers}
When we put together a real number and imaginary number, we obtain what we call a complex number.
\begin{gather*}
    \mathterm[a]{a}+ \mathterm[b]{bi}
\end{gather*}
\hspace*{4cm} Real Part\indicate{a}[out=0,in=-75]
\hspace*{1cm} \indicate{b}[out=-45,in=-90] Imaginary Part 
    
    
	\begin{tcolorbox}[title=Complex numbers,fonttitle=\bfseries,
		colframe=blue!70!red,
	    colback=white]
	A complex number is a number that can be written in the form $a+bi$, where $a$, $b$ are real 
	numbers. $a$ is called the \textbf{real part} and $bi$ is called the \textbf{imaginary part}.
	\end{tcolorbox}
	%
Examples of complex numbers include 
\begin{align*}
    3i&       &       &-10i \\
    4+6i&     &       &-1-2\sqrt{5}i\\
    \vdots\quad &   &       &\qquad \vdots
\end{align*}    
% ==== SUBSECTION
\subsection{Equality of complex numbers}
Two complex numbers $a+bi$ and $c+di$ are equal if and only if $a=c$ and $b=d$; In other words, their real parts and
their imaginary parts should be equal to each other.
%============= Example 2
\begin{exa}
Find the real numbers if $-7+2yi\sqrt{3}=x+6i\sqrt{3}$
 \begin{align*}
	\emph{Real Parts:} \qquad -7 &= x\,\, \checkmark\\
	\emph{Imaginary Parts:} \qquad 2y\sqrt{3} &= 6\sqrt{3}\longrightarrow y =3\,\, \checkmark
 \end{align*}
\end{exa}
%============= END
\begin{nt}
When performing operations (add, subtract, multilpy, divide) we can handle $i$ just
like we handle any other variable. This means when adding and subtracting complex
numbers we simply add or combine like terms.
\end{nt}
%============= Example 3
\begin{exa}
    Simplify the following expression: $(3-4i)-(-2-18i)$
\end{exa}
Be careful with negatives!
 \begin{align*}
	(3-4i)-(-2-18i)& & &\text{Distribute -1}\\
	3-4i+2+18i& & &\text{Combine like terms}\\
	5+14i& & &\text{Our Solution}
 \end{align*}

%============= END
\begin{nt}
Multiplying with complex numbers is the same as multiplying with variables with
one exception, we will want to simplify our final answer so there are no exponents
on $i$.
\end{nt}
%============= Example 4
\begin{exa}
Multiply: $(3i)(7i)$
\end{exa}
 \begin{align*}
	(3i)(7i)& & &\!\!\! \text{Multiply coefficients and $i$}\\
	21i^2& & &\!\!\! \text{Simplify, $i^2=-1$}\\
	21(-1)& & &\!\!\! \text{Multiply}\\
	-21& &  &\!\!\!\text{Our Solution}
 \end{align*}

%============= Example 5
\begin{exa}
    Multiply: $(5i)(3i-2)$
\end{exa}
 \begin{align*}
	(5i)(3i-2)& & &\text{Distribute $5i$ by $3i-2$}\\
	15i^2-10i& & &\text{Simplify, $i^2=-1$}\\
	15(-1)-10i& & &\text{Multiply}\\
	-15-10i& &  &\text{Our Solution}\\
 \end{align*}
%============= Example 6
\begin{exa}
 Use the FOIL method to multiply: $(4-2i)(3-7i)$
\end{exa}
 \begin{align*}
	(4-2i)(3-7i)& & &\text{FOIL}\\
	12-28i-6i+14(i^2)& & &\text{Simplify, $i^2=-1$}\\
	12-28i-6i+14(-1)& & &\text{Multiply}\\
	12-28i-6i-14& & &\text{Combine like terms}\\
	-2-34i& &  &\text{Our Solution}\\
 \end{align*}

%============= END
\section{Conjugates}
Imagine we have a complex number $a+bi$. Then a complex number obtained by changing the sign of imaginary part of the complex number is called the conjugate of $a+bi$.\\
Conjugates are very important. If we multiply $a+bi$ by its conjugate $a-bi$, we obtain a real number.
\begin{align*}
    (a+bi)(a-bi) &= (a)(a)+(a)(-bi)+(bi)(a)+(bi)(-bi) \\
    &=a^2-\cancel{abi}+\cancel{abi}-b^2i^2\quad 
    (\text{we know}\ \ i^2=-1)\\
    &=a^2-b^2(-1) =a^2+b^2
\end{align*}
	\begin{tcolorbox}[title=Conjugate of a complex number,fonttitle=\bfseries,
	colframe=blue!70!red,
	colback=white]
    The complex conjugate of the number $a+bi$ is $a-bi$, and the complex conjugate of $a-bi$ is $a+bi$. The multiplication of complex conjugates yields a real number
    \begin{equation*}
        (a+bi)(a-bi) = a^2+b^2
    \end{equation*}
	\end{tcolorbox}
\subsection{Rationalizing complex numbers}
It is considered a bad practice o have a imaginary number in the denominator of a fraction. Because the goal of the division procedure is to obtain a real number in the number. To rationalize a fraction, or write it in a standard form, you should multiply the numerator and denominator of division by the complex conjugate of the denominator.
 %============= Example 8
\begin{exa}
    Divide and then write it the following expression in the standard form:
    \[ \frac{4+2i}{3+4i} \]
\end{exa}
 \begin{align*}
	\frac{4+2i}{3+4i}& & &\text{Multiply by conjugate of denominator}\\
	\frac{4+2i}{3+4i}\left(\frac{3-4i}{3-4i} \right)& & &\text{Multiply}\\
	\frac{12-16i+6i-8i^2}{9+16}& & &\text{$i^2=-1$}\\
 	\frac{12-16i+6i+8}{9+16}& & &\text{Combine like terms in numerator}&&&&\\
 	\frac{20-10i}{25}& & &\text{Divide numerator by 25}&&&&\\
 	\frac{20}{25}-\frac{10}{25}i& & &\text{Simplify}&&&&\\
 	\frac{4}{5}-\frac{2}{5}i& & &\text{Our solution in the standard form}&&&&
 \end{align*}
% ========== EXAMPLE 9
\begin{exa}
    Divide and write it the following expression in the standard form:
    \[ \frac{5i}{7+i} \]
\end{exa}
 \begin{align*}
	\frac{5i}{7+i}& & &\text{Multiply by conjugate of denominator}\\
	\frac{5i}{7+i}\left(\frac{7-i}{7-i} \right)& & &\text{FOIL}\\
	\frac{35i-5i^2}{49+1}& & &\text{$i^2=-1$}\\
 	\frac{5-35i}{50}& & &\text{Divide the numerator by 50}\\
 	\frac{5}{50}-\frac{35}{50}i& & &\text{Simplify}\\
 	\frac{1}{10}-\frac{7}{10}i& & &\text{Our solution in the standard form}
 \end{align*}
